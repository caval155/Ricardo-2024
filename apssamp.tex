% ****** Start of file apssamp.tex ******
%
%   This file is part of the APS files in the REVTeX 4.2 distribution.
%   Version 4.2a of REVTeX, December 2014
%
%   Copyright (c) 2014 The American Physical Society.
%
%   See the REVTeX 4 README file for restrictions and more information.
%
% TeX'ing this file requires that you have AMS-LaTeX 2.0 installed
% as well as the rest of the prerequisites for REVTeX 4.2
%
% See the REVTeX 4 README file
% It also requires running BibTeX. The commands are as follows:
%
%  1)  latex apssamp.tex
%  2)  bibtex apssamp
%  3)  latex apssamp.tex
%  4)  latex apssamp.tex
%
\documentclass[%
 reprint,
%superscriptaddress,
%groupedaddress,
%unsortedaddress,
%runinaddress,
%frontmatterverbose, 
%preprint,
%preprintnumbers,
%nofootinbib,
%nobibnotes,
%bibnotes,
 amsmath,amssymb,
 aps,
%pra,
%prb,
%rmp,
%prstab,
%prstper,
%floatfix,
]{revtex4-2}

\usepackage{graphicx}% Include figure files
\usepackage{dcolumn}% Align table columns on decimal point
\usepackage{bm}% bold math

\usepackage{xspace}
\usepackage[per-mode=symbol]{siunitx}
\newcommand{\kindex}[2]{\ensuremath{{#1}_{\scalebox{0.65}{#2}}}}
\newcommand{\U}{\textrm{U}}
\newcommand{\Uinf}{\kindex{\U}{$\infty$}\xspace}

\begin{document}

\preprint{APS/123-QED}

\title{Physics-informed refinement of transition networks for unsteady lift prediction}% Force line breaks with \\
\thanks{A footnote to the article title}%

\author{Ricardo Cavalcanti Linhares}
 \altaffiliation[Also at ]{Physics Department, XYZ University.}%Lines break automatically or can be forced with \\
\author{other authors}%
 \email{Second.Author@institution.edu}
\affiliation{%
 University of Minnesota\\
 Aerospace engineering and mechanics
}%

\collaboration{MUSO Collaboration}%\noaffiliation

\author{Charlie Author}
 \homepage{http://www.Second.institution.edu/~Charlie.Author}
\affiliation{
 Second institution and/or address\\
 This line break forced% with \\
}%
\affiliation{
 Third institution, the second for Charlie Author
}%
\author{Delta Author}
\affiliation{%
 Authors' institution and/or address\\
 This line break forced with \textbackslash\textbackslash
}%

%\collaboration{CLEO Collaboration}%\noaffiliation

\date{\today}% It is always \today, today,
             %  but any date may be explicitly specified

\begin{abstract}
(APS abstract for now - update with a new one!! - Write it when everything else is done)
The prediction of lift during in-flight conditions poses significant challenges, and the development of effective tools for predicting aerodynamic loads is of great importance. 
In the current investigation, a data-driven approach is used to predict the lift generated by a NACA0015 airfoil under highly separated flow conditions at a Reynolds number of \num{5.5e5} using a discrete set of surface and flow field sensors as input. 
The airfoil was pitched around its static stall angle,  $\kindex{\alpha}{ss} = \ang{20}$, with a pitching amplitude of \ang{8}. 
The experimental data included six cases with reduced frequencies k from 0.025 to 0.15. 
A weighted-average based transition network was applied to measurements from 36 surface pressure taps, with 20 installed on the suction side and 16 on the pressure side of the airfoil. 
Predictive results were compared against those also incorporating planar flow field data, such as vorticity levels in specific zones near the airfoil, to improve the accuracy of lift predictions. 
The degree and nature of improvement gained by expanding the inputs to the transition network will be discussed in terms of phase-averaged performance of the prediction, and performance in capturing cycle-to-cycle variations in the experimental data. 
\end{abstract}

%\keywords{Suggested keywords}%Use showkeys class option if keyword
                              %display desired
\maketitle

%\tableofcontents

\section{\label{sec:level1}Introduction}


The lift force is an important parameter that provides valuable insights into the current state of a flying aircraft.
Maintaining stability and avoiding sudden changes in lift is essential for performing successful flights.
However, various conditions during flights such as gusts, maneuvers and atmospheric conditions, can impact the lift predicted for standard cruise conditions.

One of the conditions that leads to great variations in lift is when the aircraft reaches angles of attack higher than its static stall angle.
When this condition is reached, the flow above the wings become highly separated, accompanied by the formation of vortices and abrupt changes in pressure distribution on the wing.

Successful lift prediction becomes an important tool in these situations, identifying when the aircraft experience sudden loss or increase in lift.
By making accurate predictions, these changes can be detected on time to perform the correct steps to control and bring the aircraft back to the desired state.
This approach enhances the safety and stability of the aircraft during flight, thereby reducing risks associated with high separated flows and stall conditions.

It is known that lift can be obtained by integrating the pressure around an aircraft wing.
In practice, however, only a finite amount of pressure sensors can be installed on the wing, and the amount of sensors placed needs to a reasonable number that makes it practicable their installation, so the less sensors the better for the construction step.
For that reason it is important to identify the minimum amount of sensors needed to precisely measure the lift produced by the given aircraft in order to make it practical for their installation in aircrafts.

The task of measuring lift with sparse pressure sensors becomes more challenging as the integration method does not work anymore. 
For that reason, new methods need to be explored to successfully obtain the lift with a sparse positioning of the pressure sensors.

Machine learning techniques have been show as an useful tool to predict patterns and find intricate relationships within datasets for the field of fluid mechanics (Brunton 2019). 
The challenge of reconstructing data with incomplete data have been investigated by different authors (Manohar et al. 2018 , Jared...Brunton 2018.,Xu et al. 2023,  Clark et al.2018, Clark et al.2020, Saito et al. 2019, Chen et al. 2024 ). Specifically the identification of xxxx with sparse pressure sensing have have been done by Xu et al. 2023, .....WRITE WHAT THEY FOUND and SUCCESS IN MEASURING XXXX.

Other works tried to use the physical knowledge of the problem to create constraints or loss functions to guide the predictions.
This way the knowledge of the problem physics can be incorporated into the learning process (mention papers here...Ozan and Magri 2023, Raissi et al. 2018,...).

For the prediction of lift, aerodynamical models have been a common choice due to its practicity and accuracy for lower angles of attack (Xu and lagor 2020, Brunton and Rowley 2012, Ramesh et al. 2013 ....). 
However these models present limitations when it comes to high angles of attack where there is a highly separated flow behavior.

ongoing...




Among the existing machine learning techniques, the weighted average based transition networks have been successful to make accurate predictions in scenarios with sparse pressure sensor data (daniel's and Frieder's paper here). 

In this current work, a weighted average based transition network is used to predict lift of an airfoil with high separated flow behavior during six different pitching frequencies.
Planar PIV is used to obtain vorticity information of the flow field and relate it with the loads for each given time. 
The minimum amount of sensors and the most important locations for placing them is investigated by analyzing the predictions of the WAB transition network. 





\section{Dataset and methods}

The experimental data used was obtained at the École Polytechnique Fédérale de Lausanne (EPFL), and a detailed decription of the experimental setup can be found at He et al. 2020. 
The airfoil used was a NACA0015 at a Reynolds number of \num{5.5e5}.
The airfoil was pitched around its mean static angle, $\kindex{\alpha}{s} = \ang{20}$, at an amplitude of $\pm \ang{8}$. 
Six different reduced frequencies cases were investigated, where the reduced frequency is defined as $k = \frac{\pi f c}{\Uinf}$.

36 pressure sensors were installed around the airfoil as shown in figure XX. The sensors were positioned where the largest gradients occur. 
The reference to verify if the WAB transition networks are providing good predictions was obtained by integrating the 36 pressure values distributed around the airfoil.

Planar PIV measurements were performed to obtain the vorticity flow field (should we put an image here to illustrate or show it only in the results section? 
Or indicate here one image that will also be discussed in the results section?). ( maybe mention convective time here). 
The PIV was obtained for XX seconds, and as the six cases investigated had different pitching frequencies, the range of cycles per case observed varied from one to seven (Figure XX - show the cycles  of all cases and then maybe the comparison of one cycle of each to show how the shape vary in between them). 

It is possible to observe in Figure XX( the one just shown of the cycles) that the lift behavior during the separation part does not present a standard behavior and it is possible to note a variation on the lift, where a secondary peak might occur in some periods and with different intensities. 
These sudden increase in lift was intermittent and did not presented a standard behavior. 
Due to this unsteadiness, the separated flow region is the part that expected to be more challenging to predict and the interest is verifying if we can make this predictions accurately.

\subsection{\label{sec:level2}WAB transition network}

In this current work, the method chosen to predict the lift during the pitching motion was the weighted average based transition network (cite Iacobello).
A summary of how the WAB transition network works is presented here, and the explanation will be divided in three parts: the training, the first time step and the consecutive time steps.

\subsubsection{Training}
For implementing the weighted average fbase transition network, a training dataset is required. 
From the six different reduced frequencies experiments, the training dataset consisted in five of them, leaving one for the validation of the network.
Different combinations of cases used for training and used for validation were used in order to verify the generality of the accuracy of the predictions.
The training dataset are built by getting the pressure inputs of a given time and matching them with the corresponding measured lift coefficient that was known thanks to the experiments.
Once we have all the pressure sensors at each time matched with the corresponding lift value, a trajectory is created for that reduced frequency case (Figure XX). 

Once the trajectory for that case is constructed, it is possible to obtain the most likely transition of each node.
When observing a node in this trajectory, it is possible to see that a certain node might not always be transition to the same node (N\_next1).
The probability transition matrix will be created by taking all the times the node being investigated transitioned to N\_next1 over all possible transitions occurred on during the creation of that trajectory.
A simple example for clarity would be if the trajectory formed shows that Node x1 transitioned two times to node x2 and one time to node x3, the most likely transition of node x1 would be x2, with a 66.7\% chance.
This way, a transition probability matrix is obtained for each trajectory and saved for being used during the prediction process.

Once the training is finished, a combination of trajectories will be formed in the feature space and a transition probability matrix will be created for each trajectory.

\subsubsection{Initialization}
Once the training of the network is completed the new case lift prediction can be investigated. 
\subsubsection{Consecutive time-steps}

The new case will provide the pressure inputs of the current time and the initial estimate of the lift coefficient will be set as the same value of the previous prediction. 
With this new node that consists of the pressure sensors and the lift initial estimate, the NK closest nodes from each trajectory obtained during training are selected. 
The transition probability matrix is then used to identify the most likely nodes that these selected closest nodes are transitioning to and and weighted average is performed to obtain the new prediction and substitute the initial estimate. 
(Improve this referencing images and more clearly the step by step). 
For this current work the training consists of five different pitching frequency and the sixth case is used for testing the prediction.

show here the trajectories
explain the closest nodes when finding a new trajectory

\subsection{Quasi-steady lift aerodynamic model}

In the literature it is possible to obtain different aerodynamic models that try to predict the lift during a pitching motion. 
Even though these models present some good estimate during the attached flow region, once the airfoil reaches the stall region the predictions starts to deviate considerably from the real scenarios, and the model fails to provide accurate predictions. 
However, the information that these models can provide can be valuable as an input to the transition networks as it provides information of the pitching position. 
For this reason, the aerodynamic model used by (reference of helicopter book) is used. 
With the information about the angle of attack (alpha) and the rate of change of the angle(alpha dot), this model is able to provide some estimates of the lift.

The idea of using this models is to instead of only having the pressure inputs as a feature, we are adding the quasi-steady lift prediction as a feature. 
This way a new dimension is added in our feature space that helps differentiating the lift obtained in different regions of the pitching motion. 
(Show here a figure of the quasi-steady lift prediction compared to the integrated lift coefficient values.


\section{Results and discussion}


Show the results first with the 36 pressure sensors vs the integration of the 36 pressure sensors. 
That will be our theoretical best prediction with the WAB transition network.

Show that first some test with the transition network were performed with only the leading edge pressure sensors (decide how many, 2? all?) and show that it does not have a good prediction (check if it predict well the behavior for the non separated part.) 
then comment that the leading edge alone is not sufficient to make good predictions. 
Investigating the flow features during the separation of the flow, and then show the vorticity field for the planar PIV measurements. 
Discuss that the secondary peaks observed on the dataset are related to a region with strong vorticity above the airfoil. 
Show the pressure plot during the highest lift value of a strong secondary peak, when compared to one cycle that does not have this strong peak.
The pressure ports that present the biggest differences are the ports 13 and 14 (Check if that's for all the other cases - that's for me. 
One thing I want to try is to see if that's for all cases, or if that's working because that strong relations is valid mainly for the case that we are trying to predict.). 

As these pressure ports located at the midchord are the ones that seem to correlate better with the secondary peaks, these two ports are added to the feature space, so now we are trying to predict the lift with 4 pressure ports, two at the leading edge for the standard separation behavior and two at the mid chord. 
The results are not that bad considering that we are using only 4 ports instead of 36. 
The results capture accuratelly the lift coefficent during the pitching up motion and after the stall region the higher angles during the separation stage are well captured. 
For the secondary peak, the predictions are not so accurate but the prediction still capture the trends of the secondary peak.
In order to improve the prediction, a new feature that could provide valuable information about the pitching motion with the predicted lift was devised. 
An important flow feature would be able to add a strong correlation between the physics investigated and the value that is trying to be predicted. 
Aerodynamic models have been studied to try to predict the standard behavior during pitching motions. 
Some of them can make predictions of the quasi-steady lift coefficient by using the information of the pitching rate and the current angle of attack. 
Even though the quasi-steady lift coefficient does not capture accuratelly the lift values during the pitching down motion, when the flow is separated, the model can provide a valuable information for the feature space. 
Adding this new information to the feature space we are able to add a new dimension that can separate two different lift values that might relate to very simillar pressure sensors values. 
(what I meant here is that a set of pressure values can lead to two very different lift values behavior and the new feature helps separating them - wirte it better). 
This new feature was then added to the WAB transition network, so the feature space now consists in the 4 previous pressure sensors plus the quasi-steady model as a feature. 
The results are presented in figure XX, showing that the results are much better, and according to the RMSE (see if we will keep this metric - maybe use it only for the separated region.), the new results are even better than when we use all the 36 pressure sensors.

Show here that the quasi-steady feature could be substituted for another sensor from the bottom of the airfoil, as this sensor provides a repeatable behavior that relates the position of the pitching motion with the instant that we want to predict the lift. 
(see here if the usage of the bottom sensor improves,make it worse or is the same as using the QS model). 

It is important to mention that the quasi-steady as a feature adds the new dimension that helps separating the lift values from the pressure inputs. 
The subtraction of the quasi-steady feature from the lift values was also investigated, this way the WAB transition network would predict the variable part of the current pitching case being investigated. 
Figure XX shows that the transition network is not able to perform good predictions, as in this case we don't have a new dimension in the feature space, and we are basically using the pressure sensors on the feature space are now leading to a new value, but this corresponding value is still close to those similar pressure sensors values. 
That shows the importance of adding a new dimension.

Show cases with 2 LE, 2mid...
Show that the results are not improved if the quasi-steady curve is subtracted from the lift coefficient during the training of the WAB trans net, because it does not add a new dimension, it only changes the values to be predicted, but does not add additional information that helps predicting those values.

\section{Conclusion}

\begin{acknowledgments}
Dr. Rival and Dr. Kaiser from Queens University are gratefully acknowledged for sharing the transition weighted average transition network code. This work was supported by 
the Air Force Office of Scientific Research under AFOSR Award No .....
\end{acknowledgments}

\appendix
\pagebreak
\newpage
.
\newpage
\section{Paper outline}

\subsection{introduction}

\begin{itemize}
    \item Some motivation - importance of capturing lift during in flight conditions
    \begin{itemize}
    \item Valuable information about the current state
    \item Information can be used as inputs to control systems
    \item Leading to safer and more stable flights
    \item Add unsteady flow information
        \begin{itemize}
            \item Unsteady atmospheric conditions
            \item Separated flows
            \item Unsteady maneuvers
        \end{itemize}
    \item
\end{itemize}

\item How can we obtain this information?
    \begin{itemize}
    \item Use pressure sensors    
    \item They need to be sparse - make it easier installing etc.
    \item Best locations to position them?
    \item Because they are sparse we need to find a way to get good predictions, so which method could capture lift with not many sensors?
    \end{itemize}

    \item Introduce the method we'll be using (WAB trans. net.)
    \begin{itemize}
    \item To do what
    \item Advantages / Why we are using WAB TN
    \item Previous works that used it and performance
    \item (are we going to separate the WAB from the transition networks here in this explanation?
    \item Find here papers that used aerodynamic models to improve predictions? if not that many, say that we want to investigate that further. Mention we want to investigate how much that would improve the WAB TN method.
    \item Probably not that many will be found using aerodynamic models, so my idea is to mention a few papers that used physical modeling equations to improve the results (get some papers with that physics informed learning idea here, try to find models simillar to ours, I am not sure if it would be interesting to enter in loss functions here used to adjust the predictions).
    \end{itemize}

     \item Find gap in literature
    \begin{itemize}
    \item As we could see there were works that did .... but there is a lack of work related to...
\end{itemize}
    \item What this current work will be investigating and goals
    \begin{itemize}
    \item WAB transition network
    \item predict lift of an airfoil with high separated flow
    \item Find best locations to position sensors
    \item Using information from planar PIV to guide our decisions
    \end{itemize}    
\end{itemize}
    
    \vspace{1em}
\subsection{Dataset and methods}

\begin{itemize}
    \item Introduce the dataset
        \begin{itemize}
        \item Where it was obtained (EPFL)
        \item Write important information for current work and indicate past paper for details
        \item Airfoil; Reynolds; pitching angle amplitude; Mean angle; Reduced frequencies cases; 36 pressure sensors installed; Planar PIV measurements
        \end{itemize}

    \item Introduce the WAB transition network, how it works.

    \item Introduce the quasi-steady lift aerodynamic model
        \begin{itemize} 
        \item Explain what it is, show comparison with our measured cases
        \item Introduce the idea of QS lift as a feature that will be tested and compared with the transition networks only with pressure as the feature space.
        \end{itemize}
\end{itemize}


\subsection{Results and discussion}

\begin{itemize} 
    \item Show the predictions results with 36 pressure sensors vs the integration of the 36 pressure sensors.

    \item  Show that first attempt was trying to predict the lift only using the LE sensors, as the LE is known for the standard separation behavior.
        \begin{itemize} 
        \item  Discuss that the results are not good at all, and LE is not sufficient to make good predictions
        \item So let's investigate the flow behavior to try to understand the flow during this pitching motion 
        \end{itemize}

    \item Show relationship of the high vorticity roll up over the airfoil with the secondary peak.
        \begin{itemize}
            \item Show the plot with the pressure difference between cycles, and show that during the secondary peak the sensors that presented the biggest difference were 13 and 14 (check again if that's for all cases... or if it works because it is important for this case trying to be predicted)
        \end{itemize}
    \item Show results of the WAB TN with LE senors, but now adding the 13 and 14 that seem to be relevant
    \begin{itemize}
        \item Compare with the prediction with 36 pressure sensors and show that it is not performing that bad, considering that we just removed 32 sensors.
        \item Mention that it is capturing the trends of the peaks and secondary peaks, but for the secondary peaks it is not reaching the same magnitude when compared to the prediction with all the 36 pressure sensors.        
    \end{itemize}
    \item Add Quasi-steady aerodynamic model as a feature
        \begin{itemize}
            \item Show that now we are capturing the trends and the magnitudes are as good as with all the pressure sensors (showing a metric here like RMSE).
            \item Mention that why this feature works, as we are adding a new dimension we separate the cases that might have similar pressure inputs but end up in very different Lift coefficients.
            \end{itemize}            
    \item Show that the QS as a feature could be substituted by a pressure sensor on the bottom of the airfoil, as that is also giving information of the position of the pitching motion location on time.
    \item Discuss here that the quasi-steady lift works because it is being added as a feature, so it adds a dimension. 
    \begin{itemize}
        \item Show that subtracting QS lift from CL and trying to predict it does not work. Plot results with that.
    \end{itemize}
\end{itemize}


\subsection{Conclusion}
\begin{itemize}
    \item WAB TN had a good performance by itself when using all sensors but when we reduce them it is not so good.
        \begin{itemize}
            \item Guided by planar PIV measurements we could find the most relevant pressure sensors for capturing the patterns (not the complete magnitude)
        \end{itemize}
    \item Adding aerodynamic model as a feature is important to improve results (adds dimension), and we get results as good as when we were using all the sensors.

\end{itemize}

DONT FORGET: to show that the predictions are good for different ocmbinations of training datasets and predicted case (show generality and how performs for these different cases, does it get worse in some cases?)

Investigate low frequency vs high frequency in the training dataset. lower frequencies are more important to be there?


\newpage


\section{Papers}

\subsection{Machine learning and good ability for predicting patterns}
\begin{enumerate}
    \item Brunton et al. 2019 - Machine Learning for Fluid Mechanics
\end{enumerate}

\subsection{Machine learning predictions with sparse sensing}
\begin{enumerate}
    \item Manohar et al. 2018 - Data-Driven Sparse Sensor Placement for Reconstruction
    \item Loiseau and Brunton, 2016 - Constrained sparse Galerkin regression
    \item Kaiser...Brunton 2018 - Sparsity enabled cluster reduced-order models for control
    \item Jared...Brunton 2018. Robust flow reconstruction from limited measurements via sparse representation
    \item Clark et al.2018 - Greedy Sensor Placement With Cost Constraints
    \item Clark et al.2020 - Sensor Selection With Cost Constraints for Dynamically Relevant Bases   (Also in category C - constraints)
    \item Saito et al.2019 - Determinant-Based Fast Greedy Sensor Selection Algorithm
    \item Saito et al. 2019 - Data-driven Vector-measurement-sensor Selection based on Greedy Algorithm for Particle-image-velocimetry Measurement
    \item Xu et al. 2023 - A practical approach to flow field reconstruction with sparse or incomplete data through physics informed neural network
    \item Raynaud et al 2022 - ModalPINN: An extension of physics-informed Neural Networks with enforced truncated Fourier decomposition for periodic flow reconstruction using a limited number of imperfect sensors
    \item Chen et al. 2024 - Sparse Pressure-Based Machine Learning Approach forAerodynamic Loads Estimation During Gust Encounters
    
\end{enumerate}

\subsection{Machine learning predictions using physical loss functions/constraints (informed learning)}
\begin{enumerate}
    \item Ozan and Magri 2023 - Hard-constrained neural networks for modeling nonlinear acoustics
     \item Raissi et al. 2018 - Deep Learning of Turbulent Scalar Mixing
    \item Jagtap and Karniadakis 2019 - Adaptive activation functions accelerate convergence in deep and physics-informed neural networks
    \item Raissi et al 2019 - Physics-informed neural networks: A deep learning
    framework for solving forward and inverse problems involving nonlinear partial differential equations
    \item Ma et al. 2021 - Physics-Driven Learning of the Steady Navier-Stokes Equations using Deep Convolutional Neural Networks
\end{enumerate}

\subsection{Predictions of lift}
\begin{enumerate}
    \item Raissi et al 2018 - Deep Learning of Vortex Induced Vibrations - (with ML)
    \item Ramanujam and Ozdemir 2017 - Improving Airfoil Lift Prediction
    \item Surash et al. 2003 - Lift coefficient prediction at high angle of attack
    using recurrent neural network - (with ML)
    \item Zhang et al. 2018 - Application of Convolutional Neural Network- (with ML)
    to Predict Airfoil Lift Coefficient
    \item Matsushima et al. 2009 - Drag and Lift Prediction Based on a Wake Integration Method Using Stereo PIV
    \item Zhao et al. 2023 - Machine Learning Assisted Prediction of Airfoil Lift-to-Drag
Characteristics for Mars Helicopter  - Might not be suitable thoug, they are more concerned about making optimal parameters...
   
\end{enumerate}


\subsection{Aerodynamic models used for predictions of Lift (show limitations)}
\begin{enumerate}
    \item Xu and lagor 2020 - A New Expression for Quasi-Steady Effective Angle of Attack including airfoil kinematics and fllow field nonuniformity
    \item Brunton and Rowley 2012 - Empirical State-Space Representations
for Theodorsen’s Lift Model
    \item Ramesh et al. 2013 - An unsteady airfoil theory applied to pitching motions
validated against experiment and computation
    \item Leishman, Gordon J. Principles of helicopter aerodynamics with CD extra. Cambridge university press, 2006
    \item McGowan et al. 2011 - Investigations of Lift-Based Pitch–Plunge Equivalence
    for Airfoils at Low Reynolds Numbers
    
    \item Sivells and Neely 2013 - Method for calculating wing characteristics by lifting-line theory using nonlinear section lift data
    \item Anderson et al. 1980 - Numerical Lifting Line Theory Applied to Drooped Leading-Edge Wings Below and Above Stall
    \item Owens 1997 - Weissinger’s Model of the Nonlinear Lifting-LineMethod for Aircraft Design
    \item Spalart 2014 - Prediction of Lift Cells for Stalling Wingsby Lifting-Line Theory
    \item Cole et al. 2020 - Unsteady Lift Prediction with a Higher-Order Potential Flow Method
\end{enumerate}

\subsection{High angles of attack present challenges: dynamic stall - sudden changes}
\begin{enumerate}
    \item Mulleners and Raffel 2011 - The onset of dynamic stall revisited
    \item Gupta and Ansell 2018 - Unsteady Flow Physics of Airfoil Dynamic Stall
    \item Mulleners and Raffel 2013 - Dynamic stall development
    \item Kiefer et al. 2022 - Dynamic stall at high Reynolds numbers induced
    by ramp-type pitching motions
 \end{enumerate}


\subsection {List of other Papers to check }
\begin{enumerate}


    \item Desai and Mittal 2022- Effect of free stream turbulence on the topology of laminar separation bubble on a sphere
    \item Meena and Taira 2021- Identifying vortical network connectors for turbulent flow modification
    \item Novoa and Magri 2022 - Real-time thermoacoustic data assimilation
    \item Novoa et al.2023 - Inferring unknown unknowns: Regularized bias-aware ensemble Kalman filter (adding some bias might be important?)
    \item Iacobello et al. 2021 - A review on turbulent and vortical flow analyses via complex networks
    \item Traub 1999 - Lift Prediction of Spanwise Cambered Delta Wings
    \item Valarezo et al. 1992 - Maximum Lift Prediction for multielement wings
     \item Almohammuadi etl al. 2021 - Assessment of several modeling strategies on the prediction of lift-drag coefficients of a NACA0012  airfoil at a moderate Reynolds number
    \item Slotnick and Mavriplis 2021 - A Grand Challenge for the Advancement of Numerical Prediction of High Lift Aerodynamics
\end{enumerate}



\begin{verbatim}

\end{verbatim}

\begin{verbatim}

\end{verbatim}

% The \nocite command causes all entries in a bibliography to be printed out
% whether or not they are actually referenced in the text. This is appropriate
% for the sample file to show the different styles of references, but authors
% most likely will not want to use it.
\nocite{*}

\bibliography{}% Produces the bibliography via BibTeX.




\end{document}
%
% ****** End of file apssamp.tex ******







